\documentclass[12pt]{article}
\setlength{\parindent}{0pt}
\usepackage[top=2cm, bottom=3cm, left=3cm, right=3cm]{geometry}
\geometry{a4paper}

\usepackage{fancyhdr}

\usepackage{titlesec}
\titleformat{\section}
    {\normalfont\bfseries}
    {\thesection}{1em}{}

\usepackage{fontspec,xltxtra,xunicode}
\defaultfontfeatures{Mapping=tex-text}
\setromanfont
    [Mapping=tex-text, Ligatures={Common}]
    {Linux Libertine O}


\newcommand{\court}[1]
    {\parbox[t]{12cm}{\uppercase{#1}}}
\newcommand{\fileno}[1]
    {\parbox[t]{3cm}{\raggedleft{#1}}\\[18pt]}
\newcommand{\party}[2]
    { \court{\textbf{#1}}
      \fileno{#2} }
\newcommand{\tram}[2]{
    \vspace{-24pt}
    \hrulefill
    \vspace{6pt}
    \begin{center}
        #1
    \end{center}
    \hrulefill
}
\newcommand{\heading}[1]{
    \textbf{#1}
}
\newcommand{\citn}[2]{
    \vspace{6pt}
    \emph{#1} #2.
}
\newcommand{\para}[2]{
    \begin{quote}
    [\textbf{#1}] #2
    \end{quote}
}

% header and footer
\usepackage{fancyhdr}
\fancyhf{}
\fancyhead[LE,RO]{\vspace{1cm}\thepage} % Left side on Even pages; Right side on Odd pages
\pagestyle{fancy}
\renewcommand{\headrulewidth}{0pt}

% no section numbering
\renewcommand{\thesection}{\hspace{-1em}}

% lists
\usepackage[shortlabels]{enumitem}
\setlist{leftmargin=1cm, align=left}
\setlist[enumerate]{label*=\arabic*.}

\begin{document}

% no first page number
\pagestyle{empty}

\court{In the Federal Court of Australia \\ Western Australia district registry \\ General Division}
\fileno{WAD 005 of 2013}
BETWEEN: \\[18pt]
\party{Friends of the North West Inc}{Applicant}
and \\[18pt]
\party{Minister for the Environment, Heritage and Water}{Respondent}

\tram{
    \uppercase{\textbf{
        Respondent's outline of submissions
    }}
            \\ (Junior Counsel) \\
    }

\section{Statement of facts}

\begin{enumerate}[1.]
\item
  Petro Energy Pty Ltd proposes to build and operate a floating LNG
  plant to process gas in the Selinka Gas Field, approximately 150 km
  off the coast of WA (\textbf{Proposal}).
\item
  The \emph{Environment Protection and Biodiversity Conservation Act
  1999} (Cth) (\textbf{\emph{EPBC Act}})
\item
  On 30 July 2013, the Minister approved the proposal
  (\textbf{Decision}). e Minister's reasons for the Decision
  (\textbf{Reasons}) stated that:

  \begin{enumerate}
  \item
    in deciding to approve the taking of the action, the Minister had
    given strong consideration to a recent Commonwealth government
    policy announcement that it would `streamline' environmental
    approval of offshore gas projects and `cut environmental green tape'
    in order to ensure that the Australia offshore gas industry remained
    competitive and attractive to international investment;
  \item
    he had not delayed the decision in response to FNW's letter, as he
    considered that adequate time had been given for public comment in
    compliance with the provisions of the EPBC Act.
  \end{enumerate}
\end{enumerate}

\section{Submission 1: The Minister did not make the Decision according
to a rule or policy without regard to the merits of the particular
case.}

\begin{enumerate}[1.]
\item
  The Minister's Reasons disclose only that the Minister ``gave strong
  consideration'' to the policy. The Minister had regard to the merits
  of the
\item
  A lawful policy is normally a relevant consideration which a
  decision-maker is bound to take into account.
\item
  In the absence of any statutory or contextual indication of the weight
  to be given to factors to which a decision-maker must have regard, it
  is generally for him or her to determine the appropriate weight to be
  given to them.

  \emph{Minister for Aboriginal Affairs v Peko-Wallsend Ltd} {[}1986{]}
  HCA 40; \\(1986) 162 CLR 24, 41 (Mason J).
\item
  In Drake, the Court said:
\end{enumerate}

\begin{quote}
The propriety of paying regard to general policy considerations is most
evident in a case such as the present where there are no specified
statutory criteria for the exercise of the discretionary power and where
the power is entrusted to a Minister of the Crown responsible to
Parliament.
\end{quote}

\section{Submission 2: The Minister's refusal to delay the Decision was
reasonable in the circumstances.}

\begin{enumerate}[1.]
\item
  Section 5(2)(g) of the \emph{ADJR Act} provides that the improper
  exercise of a power includes ``an exercise of a power that is so
  unreasonable that no reasonable person could have so exercised the
  power.'' This provision incorporates the common law notion of
  `\emph{Wednesbury} unreasonableness.'

  \emph{Associated Provincial Picture Houses Ltd v Wednesbury
  Corporation} {[}1948{]} 1 KB 223, \\ 230 (Lord Green MR).

  \emph{Minister for Aboriginal Affairs v Peko-Wallsend Ltd} {[}1986{]}
  HCA 40; \\(1986) 162 CLR 24, 41 (Mason J).
\item
  The High Court considered \emph{Wednesbury} unreasonableness in
  \emph{Li}, affirming decisions at first instance and on appeal to the
  Full Court which upheld this ground of review.
\item
  This Court recently considered an application for judicial review of a
  Ministerial decision under the \emph{EPBC Act} in \emph{Tarkine
  National Coalition}. Although the application was successful, the
  Court rejected the ground of review alleging \emph{Wednesbury}
  unreasonableness, noting Gageler J's observation.

  \begin{quote}
  The above response is a persuasive one, at least in the latter two
  points. It is difficult to see how the decision could be characterised
  as irrational or so unreasonable that no decision-maker would make it.
  As Gageler J said in Li at {[}113{]} ``judicial determination of
  Wednesbury unreasonableness in Australia has in practice been rare.
  Nothing in these reasons should be taken as encouragement to greater
  frequency. This is a rare case''.
  \end{quote}
\item
  In \emph{Li}, the applicant sought judicial review of the Migration
  Review Tribunal's refusal to delay its decision on a visa application
  until the applicant could put certain evidence before the Tribunal.
  The decision was held to be ``unreasonable in the \emph{Wednesbury
  Corporation} sense'' at first instance, on appeal to the Full Court of
  the Federal Court, and on appeal to the High Court. Gageler J said:

  \begin{quote}
  Judicial determination of Wednesbury unreasonableness in Australia has
  in practice been rare. Nothing in these reasons should be taken as
  encouragement to greater frequency. This is a rare case.
  \end{quote}

  \emph{Minister for Immigration and Citizenship v Li} {[}2013{]} HCA
  18, {[}113{]} (Gageler J).
\item
  In contrast to the refusal in \emph{Li}, the Minister's refusal to
  defer the Decision was not `fatal' to the applicant's interests.

  \emph{Minister for Immigration and Citizenship v Li} {[}2013{]} HCA
  18, {[}31{]} (French CJ).
\end{enumerate}

\end{document}
